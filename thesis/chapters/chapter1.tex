W erze informacji przechowywanych w postaci elektronicznej niemożliwym wydaje się sprawna nawigacja pomiędzy różnego rodzaju tekstami bez pomocy różnego rodzaju wyszukiwarek. Umożliwiają one szybkie zawężenie dziedziny z której pochodzą dokumenty wśród których użytkownik wyszukuje te interesujące go, znacznie skracając czas niezbędny do odszukania interesujących treści.
Najbardziej powszechną wyszukiwarką jest wyszukiwarka google - dla podanego w postaci słów zapytania stara się ona znaleźć najlepiej dopasowane strony internetowe.

Celem naszego projektu jest stworzenie własnej wyszukiwarki działającej na posiadanym przez nas korpusie tekstów o tematyce finansowej. Budując nasze narzędzie wyszukiwania korzystać będziemy z reprezentacji strukturalnej artykułów. Użytkownik końcowy dostanie do ręki aplikację mobilną na platformę iOS. Artykuły na których operuje aplikacja mają pochodzić z wiodących serwisów informacyjnych o tematyce finansowej jak reuters czy bloomberg. Ze względu na zawężenie dziedziny do konkretnej tematyki możliwe jest przygotowanie narzędzi wyszukiwania bardziej wyspecjalizowanych i dostosowanych do bardziej szczegółowego problemu. 

Tworzone oprogramowanie jest oprogramowaniem o architekturze klient-serwer. Wszystkie obliczenia wykonywane być mają po stronie serwerowej a dostarczony klientowi interfejs ma jedynie służyć do przekazywania serwerowi zapytań oraz do wprowadzania ich przez użytkownika.

