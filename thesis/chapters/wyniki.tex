\section{Najważniejsze efekty pracy}
Rezultatem prac nad projektem jest w pełni funkcjonalna wyszukiwarka artykułów o tematyce finansowej. Interfejsem projektu jest aplikacja kliencka na platformę iOS, która w przyjazny i wygodny sposób umożliwia użytkownikowi wyszukiwanie interesujących go tekstów. Zgodnie z założeniami szukanie interesujących artykułów jest możliwe na dwa sposoby: za pomocą zapytania w postaci słów oraz poprzez wybieranie interesujących użytkownika kategorii. W ramach przeprowadzonych testów zauważono, że wyszukiwanie poprzez kolejne wybieranie klastrów jest wygodne a opis grup artykułów w postaci często występujących słów wystarczający by zorientować się w tematyce opisywanych tekstów. Dodatkowo poprzez prezentację kilkunastu klastrów naraz użytkownik nabiera większej orientacji w tematyce i wyczucia jakiego rodzaju artykuły potencjalnie może znaleźć - korzystając z tego rodzaju wyszukiwania możliwe jest zatem znalezienie bez podawania szczegółów już na początku procesu szukania.

\section{Potencjalne kierunki dalszych prac}
Podczas prac zauważono kilka potencjalnych kierunków dalszego rozwoju projektu:

\begin{itemize}
\item Próba zastosowania innej niż euklidesowej miary odległości podczas klasteryzacji
\item Stworzenie fragmentu aplikacji serwerowej odpowiedzialnego za dodawanie nowych artykułów do bazy potencjlanych wyników
\item Zastosowanie innego rodzaju wyboru najlepszych artykułów podczas użycia algorytmu LDA niż obliczanie iloczynów skalarnych pomiędzy reprezentacjami tematów a reprezentacją zapytania
\item Próba implementacji \textit{spelling correction} czyli próba znalezienia błędów podczas wpisywania zapytania przez użytkownika i poprawienia ich
\end{itemize}