W trakcie realizacji projektu zastosowano następujące technologie:
\begin{itemize}
  \item język Python 2.7 - interpretowany język dostępny na każdym systemie operacyjnym. Został wybrany ze względu na doświadczenie autorów, ekspresywność i ogromną liczbę bibliotek oraz rozbudowaną społeczność jego użytkowników.
 \item biblioteka LDA - biblioteka języka python. Umożliwia wygodny interfejs do obliczania LDA (ang. Latent Dirichlet Allocation) czyli probabilistycznego algorytmu redukcji wymiarów.
 \item scipy - biblioteka używana by przechowywać wektory lub macierze w postaci rzadkiej. W przypadku artykułów przechowywanych w postaci wektorów cech takie podejście skutkowało znacznym przyspieszeniem obliczeń
 
\item flask – lekka i prosta biblioteka dla języka Python umożliwiająca stawianie aplikacji webowych

\item networkx – biblioteka służąca do reprezentacji grafów. Za jej pomocą przechowywana była struktura danego artykułu podczas liczenia niezmienników grafowych. Udostępnia szybki dostęp do wielu cech zbudowanego grafu jak ilość silnie spójnych składowych, liczba wierzchołków czy krawędzi. 
 
\item język Swift 2.1 - młody język programowania (zaprezentowany przez Apple w roku 2014), służący m.in. do tworzenia aplikacji mobilnych na platformę iOS. Został wybrany ze względu 
na doświadczenie autorów oraz z uwagi na zawarte 
w nim godne uwagi koncepty i rozwiązania. 
\end{itemize}