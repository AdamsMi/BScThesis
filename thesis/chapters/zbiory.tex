Podczas obliczeń przetwarzane są dwa niezależne zbiory danych:
\section{Historyczna baza danych artyków reuters}
Korpus artykułów serwisu reuters sprzed kilku lat. Artykuły te są zbliżone tematycznie z tymi obsługiwanymi przez nas, dodatkowo są opisane przez kategorie do których należą. Postanowiliśmy więc użyć ich jako wiedzy zewnętrznej i spróbować wykorzystać więdzę którą niesie kombinacja treści artykułów i kategorii do których można je przypisać.

\section{Baza obsługiwanych artykułów}
Ponieważ celem naszej wyszukiwarki jest dostarczenie informacji o tematyce finansowej, 
posiadane przez nas artykuły są pobrane z kilku serwisów internetowych o tejże tematyce, na przykład: reuters, telegraph czy the street. Artykuły docelowo powinny być także pobierane na bieżąco, by aplikacja była atrakcyjna dla użytkownika.